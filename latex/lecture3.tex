\section{Exercises from Lecture 3}

%%%%%%%%%%%%%%%%%%%%%%%%%%%%%%%%%%%%%%%%%%%%%%%%%%%%%%%%%%%%%%%%%%%%
\subsection{Exercise 3.24}

\emph{Consider the problem}
\begin{equation}
    u^* := \argmin_{u: [0,1] \to \mathbb{R}} J(u) := \frac{1}{2} \int_0^1 \left( x(t)^2 + Pu(t)^2 \right) dt 
\end{equation}

\emph{subject to}
\begin{equation}
    \dot{x}(t) = u(t), \hspace{0.5cm} x(0) = 1, \hspace{0.5cm} x(1) = \text{free}
\end{equation}
\emph{where $P>0$ us a weighting parameter. Find an optimal solution $u^*$ and $x^*$ using the Euler-Lagrange condition.}\\
\\
\emph{Discuss how the solution changes as $P \to \infty$ and $P \to 0$. Provide an intuitive explanation of the phenomenon.}\\
\\
\textbf{Solution:}\\
\\
We can consider this problem as a fixed-time, free-endpoint problem subject to the Euler Lagrange Equations. The first step is to calculate the Hamiltonian. First, let's calculate the Lagrangian as:
\begin{equation}
    L(t, x(t), u(t)) = \frac{1}{2} \left( x(t)^2 + Pu(t)^2 \right) 
\end{equation}
The Hamiltonian will be calculated as:
\begin{align}
    H \left( t, x(t), u(t), p(t) \right) &= \langle p(t), f \left( (t), x(t), u(t) \right) \rangle - L(t, x(t), u(t)) =  \cdots \\
    \cdots &= p(t)u(t) - \frac{1}{2} x(t)^2 - \frac{P}{2}u(t)^2    
\end{align}
Now, we have to calculate the adjoint vector $p(t)$ which satisfies the following differential equation:
\begin{align}
    &\dot{p}(t) = - H_x (t,x(t), u(t), p(t)) = x(t) \\
    &\text{with } p(1) = -K_x (1, x^*(1)) = 0 \text{ (since there is no terminal cost) }
\end{align}

Therefore, $ \dot{p}(t) =x(t)$ ,with the final state condition $ p(1) = 0 $. Now, we have the first order condition on the Hamiltonian of:
\begin{equation}
    H_u(t, x(t), u(t), p(t)) = p(t) -Pu(t) = 0
\end{equation}
from which we obtain $ u(t) = \frac{p(t)}{P}$. We can summarize the influence on the state as 
\begin{equation}
    \dot{x}(t) = p(t) = Pu(t)
\end{equation}
We have the two-point boundary value problem (\emph{TPBVP}) stated as follows:
\begin{equation}
    \begin{cases}
        &\dot{x}(t) = p(t), \hspace{0.5cm} t_0 = 0, \hspace{0.5cm} x(0) =1\\
        &\dot{p}(t) = x(t), \hspace{0.5cm} t_f = 1, \hspace{0.5cm} p(1) = 0\\
    \end{cases}
\end{equation}
We can observe that $\Ddot{x}(t) = \dot{p}(t) = x(t)  \Longrightarrow \Ddot{x}(t) = x(t)$. We solve this second order differential equation whose solution is:
\begin{equation}
    \begin{cases}
        &x(t) = c_1 e^t + c_2 e^{-t} \\
        &p(t) = \dot{x}(t) = c_1 e^t - c_2 e^{-t}
    \end{cases}
\end{equation}
Now, we can obtain the values of $c_1$ and $c_2 $ from the boundary conditions:
\begin{equation}
    \begin{cases}
        &x(0) = 1 \Longrightarrow x(0) = c_1 +c_2 = 1\\
        &p(1) = 0 \Longrightarrow p(1) = c_1 e - c_2 e^{-1} \Longrightarrow c_2 = c_1 e^2
    \end{cases}
\end{equation}
from which we obtain
\begin{equation}
    \begin{cases}
        &c_1 = \frac{1}{1 + e^2}\\
        &c_2 = \frac{e^2}{1+e^2}
    \end{cases}
\end{equation}
The optimal solution then becomes:
\begin{equation}
    \begin{cases}
        &x^* (t) = \frac{1}{1 + e^2} e^t + \frac{e^2}{1+e^2} e^{-t}\\
        &p^* (t) = \frac{1}{1 + e^2} e^t - \frac{e^2}{1+e^2} e^{-t} \\
        &u^* (t) = \frac{p^*(t)}{P} = \frac{1}{P(1 + e^2)} e^t - \frac{e^2}{P(1+e^2)} e^{-t} 
    \end{cases}
\end{equation}
where $ t \in [0,1]$. We will now discuss about the influence of the weighting value $P$ on the control input.
\begin{enumerate}
    \item Case $P \to 0$:
    \begin{align}
         \lim_{P \to \infty} u^* (t) &= \lim_{P \to \infty} \frac{1}{P(1+e^2)} \left( e^t - e^2 e^{-t} \right)  = \cdots \\
         \cdots &= \lim_{P \to \infty} \frac{e^t}{P(1+e^2)} \left(1 - e^{2(1-t)} \right) = - \infty, \hspace{0.5cm} t \in [0,1]
    \end{align}
    In this case, we can deduce that the weighting $P$ going to zero makes the control input diverge: intuitively, the more the input is muffled, the larger the control input will be in its absolute value to obtain the optimal trajectory.
    \item Case $P \to \infty$:
    \begin{align}
         \lim_{P \to \infty} u^* (t) &= \lim_{P \to \infty} \frac{1}{P(1+e^2)} \left( e^t - e^2 e^{-t} \right)  = \cdots \\
         \cdots &= \lim_{P \to \infty} \frac{e^t}{P(1+e^2)} \left(1 - e^{2(1-t)} \right) = 0, \hspace{0.5cm} t \in [0,1]
    \end{align}
    In this case, we can deduce that the weighting $P$ going to infinity makes the control input go to zero: this means that, the more the input is amplified, the smaller the control input's absolute value will be to obtain the optimal trajectory.
\end{enumerate}
\QEDB

%%%%%%%%%%%%%%%%%%%%%%%%%%%%%%%%%%%%%%%%%%%%%%%%%%%%%%%%%%%%%%%%%%%%